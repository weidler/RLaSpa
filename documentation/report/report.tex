\documentclass[10pt,a4paper]{article}
\usepackage{acl2014}
\usepackage[T1]{fontenc}
\usepackage[utf8]{inputenc}
\usepackage{times}
\usepackage{url}
\usepackage{amsmath}
\usepackage[natbib=true,backend=biber,style=authoryear,maxbibnames=99]{biblatex}
\usepackage{lipsum}
\usepackage{latexsym}
\usepackage[normalem]{ulem}
\usepackage{algorithm, algorithmic}%
\usepackage[]{setspace}
\usepackage{booktabs}
\useunder{\uline}{\ul}{}
\usepackage[font=small,labelfont=bf]{caption}
\usepackage{enumerate}
\usepackage{lastpage}
\usepackage{float}
\usepackage[page]{appendix}
\pagestyle{plain} 
\usepackage{graphicx}
\usepackage{subcaption}
\usepackage{bookmark}
\title{Reinforcement Learning in Latent Space}
\date{23rd January, 2019}

\addbibresource{references.bib}
\AtBeginBibliography{\small}

\author{\normalfont Adrián Rodríguez Grillo, Danni Liu, Alessandro Scoppio,\\Kevin Trebing and Tonio Weidler\\\\
Department of Data Science and Knowledge Engineering, Maastricht University\\\today}

\setlength{\parskip}{0.5em}
\captionsetup{font=small,labelfont=bf}
\captionsetup[sub]{font=small,labelfont=bf}

\begin{document}

\maketitle

\begin{abstract}

In general, \textit{reinforcement learning} algorithms produce task-specific solutions. This makes learned knowledge unusable when learning new tasks, although there might be strong similarities between them. However, reinforcement learning often requires expensive training, raising the need for such \textit{transfer learning}. In this work, we propose a technique of guiding a Deep-Q-Network towards producing more general abstractions of visual input in order to make transfer to unseen tasks easier. For this purpose, the encoder producing the representation of the visual input is not only trained by the Deep-Q-Network performing the task, but also using multiple decoders resembling the structure of an autoencoder. Although we incorporate multiple measures to model dynamics in the representation, our results show that this task is challenging. Knowledge from learning one or multiple tasks did not successfully transfer to unseen tasks. Furthermore, an analysis of the latent space produced by the encoder network revealed the encoder's lack of focus on features relevant to solving the agent's objectives.

{{\it \bf Keywords: transfer learning, reinforcement learning, deep learning, unsupervised learning, autoencoders}}
\end{abstract}

\section{Introduction}
\label{sec:introduction}

\subsection{Research Questions}

\subsection{Approach}

\subsection{Related Work}
\label{sec:related-work}

Early work on transfer learning for reinforcement learning mostly relied on human intervention to create a mapping between source and target tasks \citep[e.g.][]{taylor2007cross}. \citet{taylor2007cross}, for example, developed a method called \textit{Rule Transfer}. Their algorithm learns a policy in the source task that gets transformed into rules, serving as advise to the agent when training in the new environment. To use these rules in the target task, hand-coded translation functions were applied. In contrast, \citet{taylor2008autonomous} published the first system that automatically mapped source and target task. They use little data from a short exploration period in the target task to approximate a one-to-many mapping between the state and action space. This is achieved by comparing all possible state-state and action-action pairs and choosing the ones with the smallest MSE when predicting the next action using neural networks trained on the target task observations. While their method effectively facilitated learning in the target task, it needs to be noted that transfer was performed on modifications of the same task. Hence, they were fairly similar and there was no attempt to tackle cross-domain transfer. 

In the following sections ...

\section{Related Work}
\label{sec:related-work}
We begin by summarizing recent work on RL with a focus on those publications that tackle tasks with visual input. Subsequently, an overview on transfer learning in the field of RL is given. 

\subsection{Reinforcement Learning for Visual Input}
Advances in Deep Learning (DL) had a great impact on the different areas of machine learning \citep{deep_learning_development}, including reinforcement learning. The main property of DL is the possibility of extracting low-dimensional representations, often called latent features, from raw or high-dimensional data like sound, images or text. This characteristic has allowed the use of RL algorithms in environments that were previously intractable due to high-dimensional or continuous states and actions.

The integration of DL methods into RL has defined the field \textit{Deep Reinforcement Learning} (DRL) whose first breakthrough came with the development of the Deep Q-Network (DQN) \citep{DQN}, an algorithm capable of playing Atari 2600 video games given visual input. This RL agent used a neural network to extract the information from the raw pixels of the screen and the game reward to learn a policy that ultimately outperformed humans in different games. The DQN approach also uses the Experience Replay \citep{replay_memory_oc} to improve the learning process. This mechanism stores the situations that the agent previously encountered as memories and randomly samples them to train the network. The sampling of memories is meant to remove correlation between training instances and to avoid forgetting previous experiences, which should facilitate learning.

Since the publication of the original DQN algorithm, many improvements have been proposed. The prioritized experience replay \citep{prioritized_memory} was introduced to consider the difference between the network prediction and the reality of the environment after the agent interaction when sampling from the memory. With the idea that some experiences are more useful than others to learn, the prioritized rexperience replay weights them with respect to this loss, prioritizing the ones that have a bigger loss.

To improve the learning process in DQNs, which may overestimate the value of the actions, the use of Double Deep Q-Networks (DDQN) \citep{DDQN} was proposed. DDQNs make use of a second network that is updated after fixed time intervals with a copy of the original network. The overestimation bias is mitigated by the introduction of the second DQN. One network functions as the actor and the other predicts the state-action pairs. Moreover, by fixing the estimations for longer periods during the training, the learning process is stabilized since the target values are not changed immediately when updating the current model.

Additionally, different architectures have been proposed to adapt to specific situations: to improve exploration \citep{noisy_dqn, hierarchical_dqn}, to benefit from different estimations \citep{DuelingDQN, distributional_dqn} and to join different improvements in one algorithm \citep{rainbow}. 

In this project we use the Double Deep Q-Network algorithm \citep{DDQN} that we adapt in order to work with an additional representation module. 

\subsection{Transfer Learning for Reinforcement Learning}
% Transfer Learning for RL
Early work on transfer learning for reinforcement learning mostly relied on human intervention to create a mapping between source and target tasks. \citet{taylor2007cross} developed a method that learns a policy on a source task that gets transformed into rules, serving as advise to the agent when training on the new environment. Hand-coded translation functions are required to use these rules in the target task.
Later \citet{taylor2008autonomous} developed the first system that automatically maps source and target task. 
They use little data from a short exploration period in the target task to approximate a one-to-many mapping between the state and action spaces.
%This is achieved by comparing all possible state-state and action-action pairs and choosing the ones with the smallest MSE when predicting the next action using neural networks trained on the target task observations. 
%While their method effectively facilitated learning in the target task, it needs to be noted that transfer was performed on modifications of the same task. 

More recent works deal with less similar source and target tasks, but still require some form of external alignment.
For example, \citet{parisotto2015actor} used an actor-mimic approach that demonstrated positive results in generalization from different Atari games.
\citet{gupta2017learning} created a latent representation for source and target tasks based on pairs of corresponding states found via time-based alignment or dynamic time warping.
%\citet{parisotto2015actor} trained an agent to learn multiple related games of the Atari Learning Environment simultaneously to later generalize from the learned experiences. The training was done by teaching the agent to mimic an expert and then doing a feature regression of the learned mimicking. 
%This can be seen as telling the agent what to do and later telling him why he should do it this way. They proposed to use this actor-mimic approach as a pre-training to increase learning speed on a set of tasks.
%\citet{gupta2017learning} used a proxy task learned in both the source and target domains, and a test task where transfer should occur. Firstly, with the proxy task, pairs of corresponding states are found using time-based alignment or dynamic time warping. Based on these state pairs, a common latent state space is learned by minimizing reconstruction errors and pairwise distances. In the test task, to incentivize policy transfer from source to task, the distance to source optimal policy in the common space is incorporated in the reward function.

More recently, model-agnostic approaches have been developed as well.
\citet{MAML} proposed a meta-learning model that is able to perform few-shot learning.
%capable of adapting to and performing in different tasks that are independent from each other.
%Using gradient descent, the meta-learner is trained over a wide range of them with the objective of constructing a general latent representation that allows the agent to behave correctly in the known activities and, also, that can be quickly adapted for new unseen tasks, using few examples.
For tasks where 
%the reward cannot be found in a few steps and exploration is needed, 
few-shot learning is challenging, this approach was extended with stochastic exploration in the later work by \citet{MAESN}.
%This algorithm uses the policy and the latent space learned in previous tasks in conjunction with noise to generate better informed exploration strategies that accelerates the learning in unseen environments.


% \citet{mnih2016asynchronous} used asynchronous gradient descent to train deep neural networks. This framework is lightweight so that it can run on a CPU instead of a GPU. They "execute multiple agents in parallel on multiple instances of the environment" \citet{mnih2016asynchronous}. This stabilized learning and reduced the training time. The reduction in training time was roughly linear to the number of processes. Asynchronous advantage actor-critic (A3C) achieved new state-of-the-art performances in 57 Atari games.

% \citet{andrychowicz2017hindsight} say that one of the biggest challenges in RL are sparse rewards. They constructed an algorithm that learns from undesired results as well as from desired results. This way the agent can learn from more experiences and thus constructing a reward function is not necessary. Constructing a good reward function is challenging (\citet{ng1999policy}) and can be complicated (\citet{popov2017data}). They showed that with their approach tasks were able to be learned that previously were not possible. Furthermore, they proposed to train an agent "on multiple goals even if we care only about one of them." \citet{andrychowicz2017hindsight}.

% RL for visual tasks/DL for RL
% PUT HERE DQN/DDQN/..., that is, general work on visual RL

% \subsubsection{Deep Q-Network}
% One of the drawbacks of table-based Q-learning occurs in environments with large state spaces.
% Maintaining and updating the values of all possible states is memory intensive and requires a great amount of training data.
% An alternative that avoids this problem is function approximation. 
% We use a Deep Q-Network (DQN) \citep{DQN}, which is a Q-learning algorithm that uses deep neural networks to approximate the state Q-values of each action.

% Experience replay is used to improve the learning process in DQN. 
% This mechanism stores the situations that the agent previously encountered as ``memories'' and randomly samples them to train the network. 
% The sampling of memories is meant to remove correlation between training instances and avoid forgetting previous experiences, facilitating the learning.



% \subsubsection{Double Deep Q-Network}
% Double Deep Q-Network (DDQN) \citep{DDQN} is an improvement to DQN that stabilizes the target Q-values to be predicted. In DQN, the maximization step taken to calculate the next Q-value (Equation \ref{eq:dqn_td}) can lead to inaccurate predictions that generates overestimation bias, affecting the learning process.

% \begin{equation}\label{eq:dqn_td}
%      Q(s,a) = r(s,a) + \gamma max_{a}Q(s',a)
% \end{equation}

% By decoupling the selection of the action from the value evaluation, as in equation \ref{eq:ddqn_td}, DDQN addresses this overestimation problem. Two networks are used in this process: the original DQN, used to choose the action that maximizes the Q-value and a target network, $Q_{t}$, that calculates the estimation with the given action.

% % This is done by the use of a target network, which will estimate the next Q-value after choosing and action with the DQN network. is updated with the trained network every certain number of steps.

% \begin{equation}\label{eq:ddqn_td}
%      Q(s,a) = r(s,a) + \gamma Q_{t}(s,argmax_{a}Q(s',a))
% \end{equation}

% More specifically, the DQN network is used and updated during the training meanwhile the target network is a snapshot of the first, that is made every certain number of steps. This procedure makes the target function fixed between updates, allowing a more stable training.


% % More specifically, the target network is kept fixed for a certain number of steps after which is updated with the DQN weights, that keeps changing every step.

% %DQN state, contrary to in DQN where it changes every step.

% \iffalse
% \begin{itemize}
% 	\item DDQN  also has prioritized memory 
% 	%Dueling DQN \citep{DuelingDQN}: state value, mix between q=learning and state action.
% 	\item DQN update policy every time. 
% 	\item continuously changing policy, estimate q-value changes every time.
% 	target network updated every 100 steps. prediction of q-values is fixed. otherwise there is ``more bias''?
% \end{itemize}
% \fi

\section{Approach}
\label{sec:approach}
We propose a modular framework for transfer learning in reinforcement learning. 
The framework consists of three modules: 
1) the \textbf{representation learner}, which constructs an abstract representation of the task environment;
2) the \textbf{policy learner}, which is a RL learner operating based on the task abstraction created by the representation learner;
3) the \textbf{task environment}, which is the RL task to be solved by the policy learner.

\subsection{Representation Learners}
The representation learner aims to construct an abstract and generalized representation of the task environment to facilitate later knowledge transfer.
Our framework includes several architectures as visualized in Figure \ref{fig:repr_learner}.
The architectures all include some lower-dimensional ``bottleneck'' layers for dimensionality reduction of the input.
Once learned, this intermediate latent space is expected to create a more compact and abstract representation of the original environment and serve as the basis of transfer learning. 

\begin{figure}[ht!]
	\centering
	\begin{subfigure}{0.45\columnwidth}
		\centering
		\includegraphics[width=\linewidth]{img/very_simple_autoencoder.pdf}
		\caption{Simple autoencoder}
		\label{subfig:repr_learner_simple_autoencoder}
	\end{subfigure}%
	~ 
	\begin{subfigure}{0.45\columnwidth}
		\centering
		\includegraphics[width=\linewidth]{img/variational_autoencoder.pdf}
		\caption{Variational autoencoder}
		\label{subfig:repr_learner_vae}
	\end{subfigure}
	\begin{subfigure}{0.5\columnwidth}
		\centering
		\includegraphics[width=\linewidth]{img/janus.pdf}
		\caption{Janus}
		\label{subfig:repr_learner_janus}
	\end{subfigure}%
	~ 
	\begin{subfigure}{0.5\columnwidth}
		\centering
		\includegraphics[width=\linewidth]{img/cerberus.pdf}
		\caption{Cerberus}
		\label{subfig:repr_learner_cerberus}
	\end{subfigure}
	\caption{Network architectures of different representation learners, where $s$ and $s'$ indicate current and next state respectively, and $a$ indicates the action leading from $s$ to $s'$. 
	The layers marked in purple are the latent representations to be used as the basis of later transfer learning. 
	The arrows indicate a straight copy from source to target.
	}
	\label{fig:repr_learner}
\end{figure}

\subsubsection{Simple and Variational Autoencoder}
One of the simplest representation learner is an autoencoder, which in our case compresses and reconstructs the current state. %representation. 
Figure \ref{subfig:repr_learner_simple_autoencoder} illustrates the simple autoencoder, and \ref{subfig:repr_learner_vae} a variational autoencoder (VAE). 
%Insert here difference of VAE and AE. 
VAE is a generative approach that models the latent distribution with a Gaussian distribution.
Whether it is simple or variational autoencoder,
one theoretical drawback of using such a simple structure for representation learning is that since the latent space only aims at compressing the state and no dynamics of the task are encoded, it is not necessarily a useful basis for transfer learning.
The upcoming architectures aim to account for this.

\subsubsection{Janus}
To create more guidance in the construction of the latent space, we add another sub-network to capture the dynamics of the environment. 
As shown in Figure \ref{subfig:repr_learner_janus},
in addition to reconstructing the current state, we also append the action in latent space and seek to use this combination to predict the next state.
The hypothesized effect of this approach is that the latent representation is incentivized to preserve transition dynamics in the environment, and therefore is a more meaningful abstraction of the task.

\subsubsection{Cerberus}
Compared to Janus, the Cerberus architecture contains one more output sub-network, which aims to predict the difference between the current and next state. 
By only predicting the differences instead of the next state in its entirety, focus is placed on the change caused by the action taken.
We expect this to be especially useful when the changes between consecutive states are small relative to the entire state representation.
The network architecture is shown in Figure \ref{subfig:repr_learner_cerberus}.  

\subsection{Policy Learners}
Based on the latent space constructed by the representation learner, the policy learner trains an RL agent to complete a given task.
%More specifically, take the output

%\subsubsection{Table-based(?)}

\subsubsection{Deep Q-Network}
One of the drawbacks of table-based Q-learning occurs in environments with large state spaces.
Maintaining and updating the values of all possible states is memory intensive and requires a great amount of training data.
An alternative that avoids this problem is function approximation. 
We use a Deep Q-Network (DQN) \citep{DQN}, which is a Q-learning algorithm that uses deep neural networks to approximate the Q-values of states.

Experience replay is used to improve the learning process in DQN. 
This mechanism stores the situations that the agent previously encountered as ``memories'' and randomly samples them to train the network. 
The sampling of memories is meant to remove correlation between training instances and facilitate learning.
Furthermore, prioritized memory \citep{prioritized_memory} is used to more frequently learn from situations that caused more loss previously. %by adjusting sampling weights.

%some memory is more important, depending on loss, give more prio to the ones that have major loss. Try to choose the ones causing more loss. 


\subsubsection{Double Deep Q-Network}
The double Deep Q-Network \citep{DDQN} is an improvement to DQN that stabilizes the target Q-values to be predicted. 
More specifically, the target is kept fixed for a certain number of steps, contrary to in DQN where it changes every step.
\iffalse
\begin{itemize}
	\item DDQN  also has prioritized memory 
	%Dueling DQN \citep{DuelingDQN}: state value, mix between q=learning and state action.
	\item DQN update policy every time. 
	\item continuously changing policy, estimate q-value changes every time.
	target network updated every 100 steps. prediction of q-values is fixed. otherwise there is ``more bias''?
\end{itemize}
\fi
The Double DQN also uses prioritized memory.

\subsection{Integrating Representation and Policy Learners}
The interaction between the representation leaner and policy learner can occur in two ways depending on the execution sequence of the two learning processes.

\begin{figure}[ht!]
	\centering
	\begin{subfigure}{\columnwidth}
		\centering
		\includegraphics[width=\linewidth]{img/history.pdf}
		\caption{Illustration of the history approach, where the representation learner is first trained and the policy learner uses the encoding produced by the trained representation learner to operate in the environment.}
		\label{subfig:approach_history}
	\end{subfigure}%
	
	\begin{subfigure}{\columnwidth}
		\centering
		\includegraphics[width=\linewidth]{img/parallel.pdf}
		\caption{Illustration of the parallel approach, where the representation and policy learners are trained simultaneously as the RL agent operates in the environment.}
		\label{subfig:approach_parallel}
	\end{subfigure}
	\caption{Illustrations of how the representation and policy learners are integrated.
	}
	\label{fig:approaches}
\end{figure}

\subsubsection{History Approach}
As the name indicates, the history approach first creates a collection of states, actions and rewards by randomly exploring in the environment.
The representation learner is then trained based on the collected history.
Once learning completes, the policy learner is trained based on the encoding of the representation learner.
In this approach, the representation and policy learning are sequential and decoupled. 

\subsubsection{Parallel Approach}
In the parallel approach, the representation and policy learners are trained simultaneously.
As the RL agent explores the environment, it chooses an action based on the encoded representation of the current state.
After executing the chosen action, the RL observes the new state and reward, which provide feedback to the representation and policy learners for their loss calculation respectively.
The loss of the policy learner is further back-propagated to the representation learner.

%\subsection{Full backpropagation(?)}

\subsection{Multi-task Learning}
\begin{itemize}
	\item structure easy to incorporate MTL
	\item every time randomly choose an environment
	\item save previously seen training instances in experience stacks
\end{itemize}
\section{Experiments}
\label{sec:experiments}

Experiments are designed to cover different modules combination of the framework and to investigate effectiveness of our approach. Specifically, each representation learner architecture is used for a variety of tasks, with the same policy module. 

\subsection{Tasks}
\label{sec:tasks}
Various tasks were explored as candidates and finding the most suitable ones.

\textbf{Classic control tasks} (citation) from OpenAI GYM are a popular benchmark for Reinforcement Learning algorithms. Despite having similar physics properties like the sinusoidal nature of the tasks, some interior aspects make them less indicated for our goal. For example, the number of features used to describe the environment changes between different tasks (e.g.: cartpole has two and mountain car has three possible actions). Furthermore, experiments from OpenAI may have different dimensions of their state representation and therefore would require a mapping or a padding.

To overcome this, tasks that have the same action space and state representation were created. They are therefore more comparable to each other and do not require a mapping of the actions or visual input. 
  \begin{enumerate}
	\item Classic control tasks
	\item Simple pathing, obstacle pathing
	\item Dynamic scroller games (tunnel, evasion)
\end{enumerate}

\paragraph{Classic control tasks}
In order to test our framework and different implementations of policy learners it was put to test on classic control tasks such as the cartpole and mountain car task. Since in their implementation the state is not visual, but physical properties such as angle speed and speed, no convolution in the representation learners are used. 

\paragraph{Pathing tasks}
To introduce visual states and to test whether the convolutional representation learners are correctly learning different pathing tasks are created. In each of these the agent needs to find a way from a starting point to an endpoint. In the simple pathing task there are no obstacles and the agent only needs to find the shortest path to the endpoint. In obstacle pathing there are obstacles blocking the agent's way and he therefore needs to go around them.

\paragraph{Scroller games}
Four different scroller games were created: Race, Evasion, Wall Evasion and Tunnel. The goal in all of these is the same: To survive by not hitting a black obstacle. The dynamics of the environment differ tho. In Race the obstacles come from the top and have the same size as the agent itself. Evasion is very similar, but the obstacles come from the right. Wall Evasion is similar to Evasion, but the obstacles are bigger, this should force the agent to plan a bit ahead when evading. Lastly, Tunnel puts the agent in the middle of a tunnel that randomly goes up or down and when the agent touches the edge of the tunnel it dies.

\subsection{Experiment Setup}

\section{Results}
\label{sec:results}

In the following we report and analyze results on the experiments described in the previous section. 

\subsection{Representation Modules}
% figures showing the different results of repr modules
% qualitative analysis

We begin by qualitatively analyzing the reconstructions created by the variational autoencoder (VAE), the Janus architecture and the Cerberus architecture.

\begin{figure}[ht!]
	\centering
	\begin{subfigure}{0.45\columnwidth}
		\centering
		\includegraphics[width=\linewidth]{"janus_recon"}
		\caption{Janus}
		\label{subfig:janus_reconstruction}
	\end{subfigure}%
	~ 
	\begin{subfigure}{0.45\columnwidth}
		\centering
		\includegraphics[width=\linewidth]{"cerberus_recon"}
		\caption{Cerberus}
		\label{subfig:cerberus_reconstruction}
	\end{subfigure}
	\begin{subfigure}{0.5\columnwidth}
		\centering
		\includegraphics[width=\linewidth]{"cvae_recon"}
		\caption{CVAE}
		\label{subfig:cvae_reconstruction}
	\end{subfigure}%
	~ 

	\caption{Reconstruction of the different representation learners. For each architecture is depicted the ground-truth (upper images) and the reconstructions (lower images).
	
	}
	\label{fig:repr_learner_reconstructions}
\end{figure}

% include the best images we have at training complete:
    % -Janus 
    % -Cerberus
    % -CVAE
    
\subsection{Policy Learner}
% tables for DQN without AE loss
    % racing
    % tunnel
    % pathing?
    
    % task | test avg | test median 

The results for the different tasks were taken from a point during a 50,000 episode training where intermediate performance was maximal. This avoids taking the policy in a state where the agent currently reconsiders its solution to potentially find a better one.

\begin{table}[]
\centering
\begin{tabular}{@{}lllll@{}}
\toprule
\textbf{Task} & \textbf{$\mu_r$} & \textbf{$\widetilde{r}$} & \textbf{episodes} & \textbf{baseline} \textbf{$\mu(r)$} \\ \midrule
Race & $2799.1$ &  & $36,000$ & 472\\
Tunnel &  &  & & 120\\ 
Pathing & -480 & -480 & 10,000 & -8988\\ \bottomrule
\end{tabular}
\caption{My caption}
\label{my-label}
\end{table}
    
% figure showing distribution of test results

\section{Discussion}
\label{sec:discussion}

% (discussion of results)

% discussion why our approach may be flawed/ what couldve been better
A possible flaw in our approach could be that the proposed representation learners do not embed useful information in the latent space. We assume that capturing the dynamics of the environment the agent is acting in is helpful, but it could be that our approach is not sufficient or practicable for the agent to use. Maybe more convolutional layers are more helpful than the handcrafted representation learners in this approach.
Furthermore, our proposed tasks could be more different to show knowledge transfer. Since we were not able to show an improvement with these similar tasks, further dissimilar tasks are not yet needed.

The underlying idea behind the autoencoder head that predicts the next state based on the encoding of the current state and the action is a reasonable idea. In the simple tasks investigated in this work though, the actions of the agent only affect its own position but not the dynamics of the environment. Furthermore, changes in the environment from one frame to another are always very small. This raises the question whether the reconstruction of the current state and the prediction of the next state are not basically the same objective, since the autoencoders can not model details precise enough to make such distinctions. This is particularly the case in multi-task learning.

% why we do not use frame overlays
    % in theory the dynamics do not need to be encoded since they are constant in the environment
    % this overlay may even be harmful since it blurs the information perceived

% discussion how our work differed from related work introduced before and why that results in different results

\subsection{Limitations}
Since our experiments do not have overwhelming results, this could be due to different factors. In the following we will address potential limitations to our results.

\paragraph{Training time} A factor concerning the bad performance of the agents on a single task could be that it was not trained long enough to find a good policy. Due to time limitations longer training was not possible.

\paragraph{Hyperparameter Tuning} Another reason for sub-optimal performance in any architecture presented in this work are non-optimal hyperparameter settings. These can be the exploration rate regarding the RL agent, but also any configuration of layer sizes and numbers for the representation module and the DDQN. Especially the latter can be crucial, since deep learning systems heavily depend on the correct setting of parameters. An exhaustive grid search was infeasible in the scope of this work, but we suggest that performance can be increased by doing a sophisticated optimization of the parameters.

\paragraph{}

\begin{itemize}
	\item representation learning for transfer learning
	\item multi-task learning
\end{itemize}
\section{Conclusion}
%\label{sec:conclusion}
%\lipsum[4]

\subsection{Contributions}
\label{subsec:contributions}
\begin{itemize}
	\item Answer to research question
	\item Other key results and findings
\end{itemize}

\subsection{Future Work}
\label{subsec:futurework}
%\lipsum[1]

{\tiny\printbibliography}

\clearpage
\raggedbottom
\appendix
\begin{appendix}
\end{appendix}

\end{document}